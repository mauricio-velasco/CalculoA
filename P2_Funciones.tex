
\documentclass[12pt, a4paper]{article}
\usepackage{hyperref}
\hypersetup{
  colorlinks=true,
  linkcolor=blue,
  urlcolor=cyan,
}
\urlstyle{same}
\usepackage[utf8]{inputenc}
\usepackage{amsmath}
\usepackage{amsfonts}
\usepackage{amssymb}
\usepackage{graphicx}


\newtheorem{theorem}{Teorema.}
\newtheorem{lemma}[theorem]{Lema.}
\newtheorem{corollary}[theorem]{Corolario.}
\newtheorem{definition}[theorem]{Definici\'on:}
\newtheorem{example}[theorem]{Ejemplo:}
\newtheorem{problema}[theorem]{Problema:}
\newtheorem{remark}[theorem]{Observaci\'on:}

\usepackage{graphicx}
\usepackage[spanish]{babel}
%\usetheme{default}

\newcommand{\pp}{\mathbb{P}}
\newcommand{\zz}{\mathbb{Z}}
\newcommand{\rr}{\mathbb{R}}
\newcommand{\qq}{\mathbb{Q}}

\usepackage{tikz, tikz-3dplot}

\definecolor{cof}{RGB}{219,144,71}
\definecolor{pur}{RGB}{186,146,162}
\definecolor{greeo}{RGB}{91,173,69}
\definecolor{greet}{RGB}{52,111,72}

\date{}

\begin{document}
\title{Pr\'actico 2: Funciones, gráficas y límites.}
\author{Mauricio Velasco}
\maketitle{}
\begin{enumerate} 
\item Dibuje en planos cartesianos distintos las regiones $(x,y)$ del plano que cumplen cada una de las siguientes condiciones:
\begin{enumerate}
\item $x>y$
\item $y<x^2$
\item $x+y$ es un entero
\item $(x-1)^2+(y-2)^3<1$
\item $x^2<y<x^4$
\end{enumerate}

\item Dibuje en planos cartesianos distintos las regiones $(x,y)$ del plano que cumplen cada una de las siguientes igualdades:

\begin{enumerate}
\item $|x|+|y|=1$
\item $|x-1|=|y-1|$
\item $xy=0$
\item $x^2-2y+4=0$
\item $x^2-y^2=0$
\item $x=|y|$
\item $x=\sin(y)$
\end{enumerate}

\item Use \verb!pyplot! para obetener dibujos de las gráficas de las funciones de abajo. Incluya su implementación y las gráficas de las funciones.  En cada una describa qué pasa cerca de cero y cuando $x$ es muy grande. 
\begin{enumerate}
\item $f(x)=x+\frac{1}{x}$
\item $g(x)=x-\frac{1}{x}$
\item $h(x)=x^2+\frac{1}{x^2}$
\item $u(x)=x^2-\frac{1}{x^2}$
\end{enumerate}

\item Mirando sólo la gráfica de la función $f(x)$ de la parte $(a)$  del ejercicio anterior, dibuje (a mano) las gráficas de las siguientes funciones:
\begin{enumerate}
\item $f(x-2)$
\item $f(x)-2$
\item $2f(x)$
\item $f(2x)$
\item $3f(3(x-1))+2$
\end{enumerate}




\item Defina la función $ReLu(x):=\max(x,0)$. Haga la gráfica de las siguientes funciones para $-2\leq x\leq 2$
\begin{enumerate}
\item $f(x)=ReLu(x)$
\item $f(x)=x+ReLu(x)$
\item $f(x)=ReLu(x)^2+x$
\item Demuestre que $|x|=ReLu(x)+ReLu(-x)$.
\end{enumerate}

\item Recuerde que una función es \emph{par} si $f(x)=f(-x)$ para todo valor de $x$ e impar si $f(-x)=-f(x)$ para todo valor de $x$. Demuestre que toda función se puede escribir de manera única como la suma de una función par y una función impar.


\item Asuma que, como vimos en clase, $\lim_{x\rightarrow 0}\frac{\sin(x)}{x}=1$ 
\begin{enumerate}
\item Implemente un programa en Python que calcule los valores de $\frac{\sin(x)}{x}$ cerca de cero (en $0.1$,$0.01$,$0.001$, etc.). Qué dicen estos números sobre el comportamiento de las funciones $\sin(x)$ y $x$ cerca de cero? Haga graficas en \verb!pyplot! que justifiquen su respuesta.

\item Usando que $\lim_{x\rightarrow 0}\frac{\sin(x)}{x}=1$ y las fórmulas para seno y coseno de la suma de dos ángulos calcule los siguientes límites explicando cuidadosamente su razonamiento:

\begin{enumerate}
\item $\lim_{x\rightarrow 0} \frac{\sin(2x)}{x}=$ 
\item Si $a,b$ son números reales calcule $\lim_{x\rightarrow 0} \frac{\sin(ax)}{\sin(bx)}=$ (el resultado depende de $a$ y $b$).
\item $\lim_{x\rightarrow 0} \frac{\sin(a+x)-\sin(a)}{x}=$ (el resultado depende de $a$) 
\item $\lim_{x\rightarrow 0} \frac{\sin(a+x)-\sin(a)}{a}=$  
\item $\lim_{x\rightarrow 0} \frac{x\sin(x)}{1-\cos(x)}=$

\end{enumerate}

\end{enumerate}



\item Dibuje la gráfica de una función cualquiera que cumpla todas las siguientes condiciones:
\begin{enumerate}
\item $f(1)=3$
\item $\lim_{x\rightarrow 1} f(x)=2$
\item $\lim_{x\rightarrow \infty} f(x)=3$
\item $\lim_{x\rightarrow -\infty} f(x)=-3$
\item $f(x)$ es impar (es decir cumple $f(-x)=-f(x)$ para todo $x$).
\item $\lim_{x\rightarrow 0^{+}} f(x)=\infty$.
\end{enumerate}

\item (\emph{Notación o-chica}) Para $a\in \mathbb{R}$ y un entero positivo $k$. Escribimos $f(x)\in o\left((x-a)^k\right)$ si y sólo si 
\[\lim_{x\rightarrow a} \frac{f(x)}{(x-a)^k} = 0.\]
(en palabras $f(x)\in o\left((x-a)^k\right)$ se lee \emph{ $f(x)$ tiene orden menor a $k$ en $a$}).
Demuestre las siguientes afirmaciones para enteros positivos $k,m,n\geq 1$

\begin{enumerate}
\item  $(x-3)^{k+1}\in o((x-3)^{k})$ y $(x-3)^{k+1}\not\in o((x-3)^{k+1})$. 
\item  Si $f(x)\in o\left((x-a)^{k+1}\right)$ entonces $f(x)\in o\left((x-a)^{k}\right)$.
\item Si $f(x)\in o\left((x-a)^{m}\right)$ y $g(x)\in o\left((x-a)^{n}\right)$ entonces:
\begin{enumerate}
\item $f(x)\cdot g(x)\in o\left((x-a)^{m+n}\right)$

\item $f(x)+ g(x)\in o\left((x-a)^{\min(m,n)}\right)$
\end{enumerate}

\end{enumerate}


\end{enumerate}



\end{document}



\item El símbolo $[x]$ denota el entero más grande $\leq x$. Dibuje la gráfica de las siguientes funciones
\begin{enumerate}
\item $f(x)=[x]$
\item $f(x)=\left[\frac{1}{x}\right]$
\item $f(x)=x-[x]$
\end{enumerate}
