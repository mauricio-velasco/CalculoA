
\documentclass[12pt, a4paper]{article}
\usepackage{hyperref}
\hypersetup{
  colorlinks=true,
  linkcolor=blue,
  urlcolor=cyan,
}
\urlstyle{same}
\usepackage[utf8]{inputenc}
\usepackage{amsmath}
\usepackage{amsfonts}
\usepackage{amssymb}
\usepackage{graphicx}


\newtheorem{theorem}{Teorema.}
\newtheorem{lemma}[theorem]{Lema.}
\newtheorem{corollary}[theorem]{Corolario.}
\newtheorem{definition}[theorem]{Definici\'on:}
\newtheorem{example}[theorem]{Ejemplo:}
\newtheorem{problema}[theorem]{Problema:}
\newtheorem{remark}[theorem]{Observaci\'on:}

\usepackage{graphicx}
\usepackage[spanish]{babel}
%\usetheme{default}

\newcommand{\pp}{\mathbb{P}}
\newcommand{\zz}{\mathbb{Z}}
\newcommand{\rr}{\mathbb{R}}
\newcommand{\qq}{\mathbb{Q}}

\usepackage{tikz, tikz-3dplot}

\definecolor{cof}{RGB}{219,144,71}
\definecolor{pur}{RGB}{186,146,162}
\definecolor{greeo}{RGB}{91,173,69}
\definecolor{greet}{RGB}{52,111,72}

\date{}

\begin{document}
\title{Pr\'actico 1: Repaso desigualdades e Inducción.}
\author{Mauricio Velasco}
\maketitle{}
\begin{enumerate} 
\item Encuentre todos los números reales $x$ que cumplen las siguientes fórmulas, justificando su respuesta

\begin{enumerate}
\item $5-x^2<8$
\item $(x-1)(x-3)>0$
\item $(x-\pi)(x+5)(x-3)>0$
\item $\frac{1}{x}+\frac{1}{1-x}>0$
\item $\frac{x-1}{x+1}>0$
\item $|x-1|+|x-2|>1$
\item $|x-1||x+2|=3$
\end{enumerate}

\item Escriba los \emph{axiomas de los números reales positivos} vistos en clase y utilice esos axiomas para demostrar las siguientes afirmaciones:
\begin{enumerate}
\item Si $a<b$ y $c>0$ entonces $ac<bc$.
\item Si $a<b$ y $c<0$ entonces $ac>bc$.
\item Si $0\leq a<b$ entonces $a^2<b^2$.
\end{enumerate}

\item Enuncie la \emph{desigualdad del triángulo} y utilícela para dar demostraciones muy cortas de las siguientes afirmaciones:
\begin{enumerate}
\item $|x-y|\leq |x|+|y|$
\item $|x|-|y|\leq |x-y|$
\item $||x|-|y||\leq |x-y|$ 
\item $|x+y+z|\leq |x|+|y|+|z|$. Para cuáles $x,y,z$ se alcanza la igualdad? 
\item $\max(x,y)=\frac{x+y+|y-x|}{2}$
\item $\min(x,y)=\frac{x+y-|y-x|}{2}$
\end{enumerate}

\item Suponga que $x,y,x_0,y_0$ son números reales cualquiera y $\epsilon>0$. 
Asumiendo que $|x-x_0|<\epsilon/2$ y $|y-y_0|<\epsilon/2$, demuestre las siguientes dos desigualdades:
\begin{enumerate}
\item $|(x+y)-(x_0+y_0)|\leq \epsilon$
\item $|(x-y)-(x_0-y_0)|\leq \epsilon$
\end{enumerate}
  
\item Suponga que $x,y,x_0,y_0$ son números reales cualquiera y $\epsilon>0$. 
\begin{enumerate}
\item Asumiendo que $|x-x_0|<\min\left(\frac{\epsilon}{2(|y_0|+1)},1\right)$ y que $|y-y_0|<\min\left(\frac{\epsilon}{2(|x_0|+1)},1\right)$, demuestre la desigualdad $|xy-x_0y_0|<\epsilon$. 

(Sugerencia: Recuerde que $xy-x_0y_0=xy-xy_0+xy_0-x_0y_0$ y factorice)

\item Interprete la desigualdad del ejercicio anterior, qué es más fácil multiplicar con precisión aproximada? números pequeños o números grandes? Justifique su respuesta.
\end{enumerate}

\item Sean $x,x_0$ números reales con $x_0\neq 0$. 
\begin{enumerate}
\item Encuentre qué debe ser $K$ para poder demostrar que  $|x-x_0|<K$ implica 
\[\left|\frac{1}{x}-\frac{1}{x_0}\right|<\epsilon\]
($K$ debe depender de $\epsilon$ y de $x_0$).

\item Interprete la desigualdad anterior. Qué es más fácil numéricamente, dividir por números grandes o por números peque\~nos? Justifique su respuesta.
\end{enumerate}

\item Demuestre que las siguientes fórmulas son válidas para cualquier entero positivo $n$:
\begin{enumerate}
\item $1^2+2^2+\dots+n^2 = \frac{n(n+1)(2n+1)}{6}$
\item $1^3+2^3+\dots+n^3 = (1+2+\dots+n)^2$
\item Implemente las fórmulas en python y verifique su validez (usando \verb!assert!) para los primeros $100$ valores de $n$ (escriba sólo el código de su implementación más no el output).
\end{enumerate} 

\item Resuelva los siguientes problemas:
\begin{enumerate}
\item
Usando el ejercicio anterior, encuentre fórmulas para las siguientes sumas: 
\begin{enumerate}
\item $1+3+5+7+\dots+ 2n-1 = ?$
\item $1^2+3^2+5^2+\dots+ (2n-1)^2 = ?$
\end{enumerate}
\item Implemente las fórmulas que encontró en Python y verifique su validez (usando \verb!assert!) para los primeros $100$ valores de $n$ (entregue solo el código de su implementación pero no el output).
\item Demuestre la validez de las fórmulas que encontró usando inducción.
\end{enumerate}


\end{enumerate}



\end{document}



