\documentclass[12pt, a4paper]{article}
\usepackage{hyperref}
\hypersetup{
  colorlinks=true,
  linkcolor=blue,
  urlcolor=cyan,
}
\urlstyle{same}
\usepackage[utf8]{inputenc}
\usepackage{amsmath}
\usepackage{amsfonts}
\usepackage{amssymb}
\usepackage{graphicx}


\newtheorem{theorem}{Teorema.}
\newtheorem{lemma}[theorem]{Lema.}
\newtheorem{corollary}[theorem]{Corolario.}
\newtheorem{definition}[theorem]{Definici\'on:}
\newtheorem{example}[theorem]{Ejemplo:}
\newtheorem{problema}[theorem]{Problema:}
\newtheorem{remark}[theorem]{Observaci\'on:}

\usepackage{graphicx}
\usepackage[spanish]{babel}
%\usetheme{default}

\newcommand{\pp}{\mathbb{P}}
\newcommand{\zz}{\mathbb{Z}}
\newcommand{\rr}{\mathbb{R}}
\newcommand{\qq}{\mathbb{Q}}

\usepackage{tikz, tikz-3dplot}

\definecolor{cof}{RGB}{219,144,71}
\definecolor{pur}{RGB}{186,146,162}
\definecolor{greeo}{RGB}{91,173,69}
\definecolor{greet}{RGB}{52,111,72}

\date{}

\begin{document}
\title{Pr\'actico 9: Técnicas básicas de integración y aplicaciones.}
\author{Mauricio Velasco}
\maketitle{}
\begin{enumerate}
\item {\it Integración por sustitución.}  Resuelva las siguientes integrales indefinidas

\begin{enumerate}
\item $\int\frac{\sin(\sqrt{x})}{\sqrt{x}}dx=$
\item $\int \sin^3\theta\cos\theta d\theta=$
\item $\int (4+x^2)^{15}dx=$
\item $\int t^2\sin(1-t^3)dt=$
\item $\int \cos(x)\cos(\sin(x))dx$
\end{enumerate}

\item {\it Resuelva las siguientes ecuaciones diferenciales separables.}
\begin{enumerate}
\item Encuentre la función diferenciable $y(x)$ que cumple $y(0)=2$ y $\frac{dy}{dx}=\frac{3x^2}{y}$.
\item Encuentre la función diferenciable $y(x)$ que cumple $y(0)=1$ y $\frac{dy}{dx}=\frac{y}{x+3}$.

\item Encuentre la función diferenciable $y(x)$ que cumple $y(-1/3)=0$ y $\frac{dy}{dx}=\frac{1+y^2}{2+3x}$.
\item Encuentre todas las funciones diferenciables $y(x)$ que cumplen $\frac{dy}{dx}=\frac{y+1}{x+1}$.
\end{enumerate}

\item Para cada ecuación diferencial del problema anterior haga una gráfica en \verb!pyplot! del campo de direcciones de la ecuación y de la solución que ud encontró en alguna región cercana a la condición inicial dada en el problema.

\item 

\end{enumerate}
\end{document}