
\documentclass[12pt, a4paper]{article}
\usepackage{hyperref}
\hypersetup{
  colorlinks=true,
  linkcolor=blue,
  urlcolor=cyan,
}
\urlstyle{same}
\usepackage[utf8]{inputenc}
\usepackage{amsmath}
\usepackage{amsfonts}
\usepackage{amssymb}
\usepackage{graphicx}


\newtheorem{theorem}{Teorema.}
\newtheorem{lemma}[theorem]{Lema.}
\newtheorem{corollary}[theorem]{Corolario.}
\newtheorem{definition}[theorem]{Definici\'on:}
\newtheorem{example}[theorem]{Ejemplo:}
\newtheorem{problema}[theorem]{Problema:}
\newtheorem{remark}[theorem]{Observaci\'on:}

\usepackage{graphicx}
\usepackage[spanish]{babel}
%\usetheme{default}

\newcommand{\pp}{\mathbb{P}}
\newcommand{\zz}{\mathbb{Z}}
\newcommand{\rr}{\mathbb{R}}
\newcommand{\qq}{\mathbb{Q}}
\newcommand{\EE}{\mathbb{E}}

\usepackage{tikz, tikz-3dplot}

\definecolor{cof}{RGB}{219,144,71}
\definecolor{pur}{RGB}{186,146,162}
\definecolor{greeo}{RGB}{91,173,69}
\definecolor{greet}{RGB}{52,111,72}

\date{}

\begin{document}
\title{Pr\'actico 8: Integración y algunas aplicaciones.}
\author{Mauricio Velasco}
\maketitle{}
\begin{enumerate}
\item Utilice una sustitución para encontrar todas las primitivas de las siguientes funciones,
\begin{enumerate}
\item $\int x^3\cos(x^4+7)dx =$
\item $\int \frac{2x}{\sqrt{1-4x^2}} =$
\item $\int \sin(5x)dx =$
\item $\int \sec^2(6\theta)d\theta=$
\item $\int \cos(x)\cos(\sin(x))dx=$
\end{enumerate}

\item Evalue todas las integrales del ejercicio anterior en el intervalo donde la variable recorre $[1,\pi]$. Acompañe cada integral con una gráfica del integrando en \verb!pyplot! en la región de interés. Marque el área que representa el valor de la integral en cada caso.

\item La tasa de producción de circuitos de una compañía esta dada por
\[y'(t) = 5000\left(1-\frac{100}{(t+10)^2}\right)\]
con $t$ medido en semanas.
\begin{enumerate}
\item Cuántos circuitos se producen desde el principio de la tercera semana hasta el final de la cuarta? 
\item Si en el instante inicial la compañia tenia $y(0)=1000$ circuitos encuentre una fórmula para el número total de circuitos disponibles en el instante $t$ para $t\geq 0$
\end{enumerate}

\item La función de Fresnel esta dada por
\[C(x)=\int_0^x\cos(\pi t^2/2)dt\]
\begin{enumerate}
\item En qué intervalos es $C(x)$ una función decreciente?
\item En qué intervalos es $C(x)$ una función convexa?
\item Explique cómo usaría el método de Newton para resolver la ecuación 
\[\int_0^x\cos(\pi t^2/2)dt = 0.7\]
\item Implemente en \verb!python! su solución del numeral anterior y encuentre una aproximación a la solución con $3$ cifras decimales correctas.
\end{enumerate}  

\item Sea $f(x)=x^2$ para $0\leq x\leq 1$ y suponga que $X$ es una variable aleaoria con $\mathbb{P}(X\in [a,b])$ proporcional a $\int_a^b f(x)dx$ para todos $a,b$.
\begin{enumerate}
\item Encuentre la función de densidad de $X$.
\item Cuál es la probabilidad de que $X$ sea menor que $1/2$?
\item Encuentre una fórmula para el valor esperado de $X$.
\item Encuentre una fórmula para $\EE[X^k]$ para cada entero positivo $k$. Calcule el valor de la varianza de $X$.
\end{enumerate}

\item Una variable aleatoria $X$ tiene distribución exponencial con parámetro dado $\lambda>0$ si su densidad esta dada por la fórmula
\[f(x)=\begin{cases}
\lambda \exp(\lambda x)\text{ , si $x>0$}\\
0\text{ , de lo contrario.}
\end{cases}
\]
\begin{enumerate}
\item Demuestre que $f(x)$ es una distribución de probabilidad.
\item Demuestre que el valor esperado de $X$ es $1/\lambda$.
\item Calcule la varianza de $X$.
\end{enumerate}

\end{enumerate}

 
\end{enumerate}

\end{document}