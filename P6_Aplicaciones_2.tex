
\documentclass[12pt, a4paper]{article}
\usepackage{hyperref}
\hypersetup{
  colorlinks=true,
  linkcolor=blue,
  urlcolor=cyan,
}
\urlstyle{same}
\usepackage[utf8]{inputenc}
\usepackage{amsmath}
\usepackage{amsfonts}
\usepackage{amssymb}
\usepackage{graphicx}


\newtheorem{theorem}{Teorema.}
\newtheorem{lemma}[theorem]{Lema.}
\newtheorem{corollary}[theorem]{Corolario.}
\newtheorem{definition}[theorem]{Definici\'on:}
\newtheorem{example}[theorem]{Ejemplo:}
\newtheorem{problema}[theorem]{Problema:}
\newtheorem{remark}[theorem]{Observaci\'on:}

\usepackage{graphicx}
\usepackage[spanish]{babel}
%\usetheme{default}

\newcommand{\pp}{\mathbb{P}}
\newcommand{\zz}{\mathbb{Z}}
\newcommand{\rr}{\mathbb{R}}
\newcommand{\qq}{\mathbb{Q}}

\usepackage{tikz, tikz-3dplot}

\definecolor{cof}{RGB}{219,144,71}
\definecolor{pur}{RGB}{186,146,162}
\definecolor{greeo}{RGB}{91,173,69}
\definecolor{greet}{RGB}{52,111,72}

\date{}

\begin{document}
\title{Pr\'actico 6: Ecuaciones diferenciales.}
\author{Mauricio Velasco}
\maketitle{}
\begin{enumerate}
\item Considere la ecuación diferencial $y'=\frac{y^2-1}{2}$.
\begin{enumerate}
\item Demuestre que para todo valor de $c\in \mathbb{R}$ la función $y(t)=\frac{1+ce^t}{1-ce^t}$ es solución de la ecuación diferencial.
\item Haga una sola gráfica en \verb!pyplot! que contenga las gráficas de  $y(t)$ para $-5\leq t\leq 5$ para $5$ valores distintos de la constante $c$ escogidos por ud (el dibujo debe ser legible y aclarar el valor de $c$ en \verb!legend!).
\item Encuentre una solución del problema de valor inicial
\[
\begin{cases}
y'=\frac{y^2-1}{2}\\
y(0)=1
\end{cases}
\]
\end{enumerate}
\item Considere la ecuación diferencial $y''+2y'+y=0$.
\begin{enumerate}
\item Cuáles de las siguientes funciones son soluciones de la ecuación diferencial? Justifique regurosamente su respuesta
\begin{enumerate}
\item $y=e^t$
\item $y=e^{-t}$
\item $y=te^t$
\item $y=t^2e^{-t}$
\end{enumerate}
\item Demuestre que si $f(t)$ y $h(t)$ son soluciones de la ecuación diferencial tambien lo es $Af(t)+Bh(t)$ para cualquier par de constantes $A$ y $B$.
\item Utilice las partes $(a)$ y $(b)$ del ejercicio para encontrar una solución del problema de valor inicial
\[
\begin{cases}
y''+2y'+y=0\\
y(1)=e\\
y'(1)=e\\
\end{cases}
\]
\end{enumerate}
\item Haga gráficas en \verb!pyplot! de los campos de direcciones de las siguientes ecuaciones diferenciales:

\begin{enumerate}
\item $4y'+12y=60$. Qué sucede con las soluciones cuando la variable independiente $x$ aumenta?
\item $y'=y^3-4y$. Segun el dibujo, para qué valores de la constante $c$ existe el límite $\lim_{x\rightarrow \infty}y(x)$ para una solución con de $y(0)=c$? Cuáles son los valores posibles de este límite? Justifique sus respuestas.
\end{enumerate}

\item Realice los siguientes pasos:
\begin{enumerate}
\item Haga una implementación  del método de Euler en python. Escriba el código de la misma.
\item Utilice su implementación para calcular soluciones numéricas de los siguientes problemas de valor inicial para $h\in \{0.1,0.01,0.001\}$ y $0\leq x\leq 5$:
\begin{enumerate}
\item 
\begin{cases}
$y'=x^2+y$\\
$y(1)=1$
\end{cases}
\item
\begin{cases}
$y'=y(4-y)$\\
$y(0)=1$\\
\end{cases}
\end{enumerate}

\end{enumerate}



\end{enumerate}



\end{document}
