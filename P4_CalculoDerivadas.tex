
\documentclass[12pt, a4paper]{article}
\usepackage{hyperref}
\hypersetup{
  colorlinks=true,
  linkcolor=blue,
  urlcolor=cyan,
}
\urlstyle{same}
\usepackage[utf8]{inputenc}
\usepackage{amsmath}
\usepackage{amsfonts}
\usepackage{amssymb}
\usepackage{graphicx}


\newtheorem{theorem}{Teorema.}
\newtheorem{lemma}[theorem]{Lema.}
\newtheorem{corollary}[theorem]{Corolario.}
\newtheorem{definition}[theorem]{Definici\'on:}
\newtheorem{example}[theorem]{Ejemplo:}
\newtheorem{problema}[theorem]{Problema:}
\newtheorem{remark}[theorem]{Observaci\'on:}

\usepackage{graphicx}
\usepackage[spanish]{babel}
%\usetheme{default}

\newcommand{\pp}{\mathbb{P}}
\newcommand{\zz}{\mathbb{Z}}
\newcommand{\rr}{\mathbb{R}}
\newcommand{\qq}{\mathbb{Q}}

\usepackage{tikz, tikz-3dplot}

\definecolor{cof}{RGB}{219,144,71}
\definecolor{pur}{RGB}{186,146,162}
\definecolor{greeo}{RGB}{91,173,69}
\definecolor{greet}{RGB}{52,111,72}

\date{}

\begin{document}
\title{Pr\'actico 4: Cálculo de derivadas.}
\author{Mauricio Velasco}
\maketitle{}
\begin{enumerate}
\item {\bf Definiciones básicas.} Complete los seguientes enunciados con definiciones correctas y completas:
\begin{enumerate}
\item La función $f(x)$ es continua en $x=a$ si...
\item El valor de la derivada de la función $h(x)$ en el punto $x=8$ es el valor del siguiente límite...
\item La interpretación geométrica del número $f'(a)$ es que este es igual a la pendiente de...
\end{enumerate}


\item Sea $f(x)=1/x$. 
\begin{enumerate}
\item Encuentre una fórmula para $f'(a)$ a partir de la definición de la derivada como un límite.
\item Encuentre la ecuación de la recta tangente a la gráfica de $f$ en el punto $(a,1/a)$.
\item Demuestre que para todo número $a$, la recta tangente del punto anterior intersecta a la gráfica de $f$ sólamente en el punto $(a,1/a)$. 
\end{enumerate}
\item Sea $T(x)=1/\sqrt{x}$
\begin{enumerate}
\item Encuentre una fórmula para $T'(a)$ a partir de la definición de la derivada como un límite asumiendo $a>0$
\item Qué puede decir sobre $T'(0)$? Haga una gráfica en \verb!pyplot! que acompañe y justifique su respuesta.
\end{enumerate}

\item Demuestre, usando inducción matemática y la regla del producto que la derivada del polinomio $f_k(x)=x^k$ en $a$ es $ka^{k-1}$ para todo entero positivo $k$.

\item Demuestre, usando la definición de derivada, que para todo par de funciones $f(x),g(x)$  diferenciables en $x=a$ y para todo número real $c$ se tiene que:
\begin{enumerate}
\item $\left(f(x)+g(x)\right)'(a) = f'(a)+g'(a)$
\item $\left(cf(x)\right)'(a)= cf'(a)$.
\end{enumerate} 



\item Calcule una fórmula para $f'(x)$ en los siguientes casos y encuentre la función lineal $\ell(x)$ que mejor aproxima a $f(x)$ cerca de $x=0.5$ (puede usar todas las reglas de diferenciación que vimos en clase). En cada caso, usando \verb!pyplot! haga un dibujo de la funci\'on y de la recta con pendiente $f'(0.5)$ que pasa por $(0.5,f(0.5))$.
\begin{enumerate}
\item $f(x)=\sin(x+x^2)$
\item $f(x)=\sin(x)+\sin(x^2)$
\item $f(x)=\sin(\sin(x))$
\item $f(x)=\sin(x+\sin(x))$
\end{enumerate}


\item Calcule una fórmula para $f'(x)$ en los siguientes casos y encuentre la función lineal $\ell(x)$ que mejor aproxima a $f(x)$ cerca de $x=0.5$ (puede usar todas las reglas de diferenciación que vimos en clase). En cada caso, usando \verb!pyplot! haga un dibujo de la funci\'on y de la recta con pendiente $f'(0.5)$ que pasa por $(0.5,f(0.5))$.
\begin{enumerate}
\item $f(x)=\sin((x+1)^2(x+2))$
\item $f(x)=\sin^2(x)\sin(x^2)$
\item $f(x)=(x+\sin^5(x))^6$
\item $f(x)=\frac{\sin(x)}{1+\sin^2(x)}$
\end{enumerate}
\end{enumerate}



\end{document}
